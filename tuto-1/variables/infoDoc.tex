% Classe
% 1: PTSI				6: PSI*			11: TSI2		16: Spé
% 2: PT	(par défaut)	7: MPSI			12: ATS
% 3: PT*				8: MP			13: PC
% 4: PCSI				9: MP*			14: PC*
% 5: PSI				10: TSI1		15: Sup
%\newcommand{\UPSTIidClasse}{2}

% Affichage personnalisé de la classe
\newcommand{\UPSTIclasse}{TUK}
\newcommand{\UPSTIvariante}{8}
% Matière
% 1: S2I (par défaut)    2: IPT     3: TIPE
% 6: Vie au lycée
\newcommand{\UPSTIidMatiere}{0}
\newcommand{\UPSTIintituleMatiere}{Robotic Workshop}
\newcommand{\UPSTIsigleMatiere}{}
% Type de document
% 0: Custom*				7: Fiche Métho de			14: Document Réponses
% 1: Cours (par défaut)		8: Fiche Synthèse    		15: Programme de colle
% 2: TD     				9: Formulaire
% 3: TP						10: Memo
% 4: Colle					11: Dossier Technique
% 5: DS						12: Dossier Ressource
% 6: DM						13: Concours Blanc
% * Si on met la valeur 0, il faut décommenter la ligne suivante:
%\newcommand{\UPSTItypeDocument}{Custom}
\newcommand{\UPSTIidTypeDocument}{3}

% Titre dans l'en-tête
% Titre dans l'en-tête

\newcommand{\UPSTItitreEnTete}{Robotic Workshop}
% \newcommand{\UPSTItitreEnTetePages}{UPSTItitreEnTetePages}
% \newcommand{\UPSTIsousTitreEnTete}{UPSTIsousTitreEnTete}


% Titre
%\newcommand{\UPSTItitrePreambule}{UPSTItitrePreambule}
\newcommand{\UPSTItitre}{Testing sensors}

% Durée de l'activité (pour DS, DM et TP)
\newcommand{\UPSTIduree}{2 h}

% Note de bas de première page
\newcommand{\UPSTInoteBasDePremierePage}{G. Vaquette}
% Numéro (ajoute " n°1" après DS ou DM)
%\newcommand{\UPSTInumero}{}

% Numéro chapitre
%\newcommand{\UPSTInumeroChapitre}{}

% En-tête customisé
%\newcommand{\UPSTIenTetePrincipalCustom}{UPSTIenTetePrincipalCustom}

% Message sous le titre
%\newcommand{\UPSTImessage}{Message sous le titre}
